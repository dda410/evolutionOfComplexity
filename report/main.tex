% This is samplepaper.tex, a sample chapter demonstrating the
% LLNCS macro package for Springer Computer Science proceedings;
% Version 2.20 of 2017/10/04
%
\documentclass[runningheads]{llncs}
%
\usepackage{graphicx}
\usepackage[toc,page]{appendix}
\usepackage{listings}
\usepackage{color}
\lstset{
  basicstyle=\ttfamily,
  columns=fullflexible,
%  frame=single,
  breaklines=true,
}

% Used for displaying a sample figure. If possible, figure files should
% be included in EPS format.
%
% If you use the hyperref package, please uncomment the following line
% to display URLs in blue roman font according to Springer's eBook style:
% \renewcommand\UrlFont{\color{blue}\rmfamily}

\begin{document}
%
\title{Cooperative and Selfish Group Evolvability}
%
%\titlerunning{Abbreviated paper title}
% If the paper title is too long for the running head, you can set
% an abbreviated paper title here
%
\author{Dimitri Diomaiuta}
%
% \authorrunning{F. Author et al.}
% First names are abbreviated in the running head.
% If there are more than two authors, 'et al.' is used.
%
\institute{University of Southampton}
%% \institute{Princeton University, Princeton NJ 08544, USA \and
%% Springer Heidelberg, Tiergartenstr. 17, 69121 Heidelberg, Germany
%% \email{lncs@springer.com}\\
%% \url{http://www.springer.com/gp/computer-science/lncs} \and
%% ABC Institute, Rupert-Karls-University Heidelberg, Heidelberg, Germany\\
%% \email{\{abc,lncs\}@uni-heidelberg.de}}
%
\maketitle              % typeset the header of the contribution
%
\begin{abstract}
In this paper we analyze and reproduce the experiments discussed in
the paper \textit{Individual
Selection For Cooperative Group Formation} \cite{groups}. In the first
part of this paper we discuss and reimplement the model described in
\cite{groups} that aims at evolving cooperative and selfish traits. In
the second part we extend the paper by applying the niche construction
process to the time spent reproducing in groups parameter.
%\keywords{First keyword  \and Second keyword \and Another keyword.}
\end{abstract}
%
%
%

\section{Introduction}
The biological inspiration of the research is bacteria growth. In this
biological scenario we can observe the different types of bacteria,
that under limited resources they behave in a cooperative way (growing
slower and consuming less) or in a selfish one (growing faster and
consuming more resources).
The originality of \textit{T. Powers} research \cite{groups} lies in
the fact that cooperation is not favoured by environmental
conditions. Techniques like spatially structured population, described
in \cite{b1,b2,b3,b5}, or not sharing resources with the rest of the
population are avoided. Another approach that is avoided in the
paper is the trait group aggregation shown in \cite{b6,b7,b8}. Trait
group aggregation favours the cooperators by exploiting the
differential group productivity. Other environmental settings that
facilitates the growth of the cooperative trait is shown in
\cite{b7,b8,b9}. Here the individuals with similar genetic traits are
grouped together, promoting the cooperative trait. The main research
point of the paper is, hence, to observe which genetic trait will
emerge when environmental conditions that favours cheaters or
cooperator are evolved as well.

The technique used to evolve the environmental settings is the niche
construction \cite{b10}. The model is composed by individuals that
specify two genetic traits:
\begin{itemize}
\item Strategy trait: cooperative or selfish
\item Group size trait: large or small
\end{itemize}
Following the settings of the model we can see that there are four
different kind of individuals: \textit{Cooperative + Large},
\textit{Cooperative + Small}, \textit{Selfish + Large} and
\textit{Selfish + Small}.
The model settings favours large groups, having a larger assigned
resource, and selfish ones, having a faster growth rate. This may to
\textit{Selfish + large} to win over the other species. Instead, when
all the four possible individual types are present, we can observe
\textit{Cooperative + Small} reaching fixation.

\section{Model Details}
We describe in this section the settings of the model we
reproduced. As Powers describes \cite{groups} the model has a
population of size N where the 4 types of individuals have, initially,
an equal distribution. The strategy genotype specifies the amount of
resources that and individual consumes and its growth rate. The group
size genotype specifies the type of group that the individual will be
part of. After the initialization phase the main routine of the
program starts. In this routine the individuals of the initial
population are in one group, called migrant pool. The second phase,
called aggregation, corresponds to randomly divide individuals into
groups according to their group size genotype, discarding the
unassigned individuals. The third phase is about performing reproduction
within the groups, as described in subsection \ref{repr}. In Powers'
model reproduction occurs only within groups for $t$ generations. After the reproduction
step the offspring joins again the migrant pool, this is the fourth
step. The last step is about rescaling the population back to its
original size N. This is done by maintaining the proportion between
the individuals. This routine continues for $T$ generations.

\subsection{reproduction step}\label{repr}
Reproduction, in this model, depends on $R$, a limited resource influx
that each group is given. The amount of resource depends on the group
size. Large groups have a 5\% per capita more resource than their
smaller counterpart. Reproduction, in this model, happen asexually and
the offspring are not affected by mutation. The offspring size for
each individual type is determined only by the consumption and growth rate
of their parents given a limited resource influx and the distribution
of other individuals inside their group. The amount of $R$ that
each genotype receives is given by equation \ref{rshare}.
\begin{equation}\label{rshare}
r_i = \frac{n_i G_i C_i}{\sum_{j} (n_j G_j C_j)}
\end{equation}
The genotype $i$ receives $r_i$ resource share. $n_i$ represents the
number of individuals of genotype $i$ in the current generation and
$G_i$ and $C_i$ represents respectively the growth and consumption
rate. At the denominator, genotype $j$ represents all the other
genotypes, $i$ included. Given the resource share of a genotype for a
given group, its offspring size is determined by the replicator equation \ref{offspring}.
\begin{equation}\label{offspring}
n_i(t + 1) = n_i(t) + \frac {r_i}{C_i} - Kn_i(t)
\end{equation}
The current generation is represented by $t$. The replicator equation
\ref{offspring} introduces also a constant mortality term represented by $K$.

\section{Implementation reproduction}
In this section we discuss the reimplementation of Powers' research
\cite{groups} and the results we obtained.

\subsection{Implementation details}
The first step towards reproduction was to gather all the parameters
used throughout the paper. We were able to reproduce the paper using
the exact same parameter used by Powers. The parameters that were
missing from the original paper, like the resource share assigned to
the groups and the death rate, were given by the University of
Southampton through assignment specifications. Table
\ref{tab_parameters} contains all the parameters used for our
reimplementation.
\begin{table}
\caption{The parameters settings of our reproduced model.}\label{tab_parameters}
\begin{tabular}{|l|r|}
\hline
\textbf{Parameters}~~~ & \textbf{Value}~~~ \\
\hline
Growth rate (cooperative), $G_c$ & 0.018 \\
\hline
Growth rate (selfish), $G_s$ & 0.02 \\
\hline
Consumption rate (cooperative), $C_c$ & 0.1 \\
\hline
Consumption rate (selfish), $C_s$ & 0.2 \\
\hline
Population size, $N$ & 4000 \\
\hline
Number of generations, $T$ & 1000 \\
\hline
Small group size & 4 \\
\hline
Large group size & 40 \\
\hline
Small group resource & 4 \\
\hline
Large group resource & 50 \\
\hline
Number of generations within group $t$ & 4 \\
\hline
Death rate $K$ & 0.1 \\
\hline
\end{tabular}
\end{table}
The program we reimplemented consists of four Java classes. The
\textit{Individual} object (see code at appendix \ref{ind}) contains the two
genotypes and is used to represents the bacteria. The
\textit{IndividualFrequencies} object (see code at appendix \ref{indfre})
represents the frequencies for a given individual type (among the four
possible types). The \textit{GroupFrequencies} (see code at appendix
\ref{groupfre}) describes the frequencies of all the four types for a
given aggregation. Lastly, the \textit{Main} class of the program (see
code at appendix \ref{main}) is responsible for the aggregation in
groups, the reproduction, the dispersal and the
normalization steps. We reimplemented the resource allocation
(\ref{rshare}) and the replicator equation (\ref{offspring}) exactly
how reported in the original paper. The only relevant implementation
choice that we took independently from the original paper, due to not
appearing in Powers' paper, was the rounding of the offspring. Since
the replicator equation (\ref{offspring}) was not producing integers,
after some testing, we decided to round the offspring during the
normalization step.

\subsection{Results}
In this section we analyze the results obtained. Figure \ref{fig1}
summarizes the obtained results. The code that produced figure
\ref{fig1} can be found in appendix \ref{plot1}.
\begin{figure}
\includegraphics[width=\textwidth]{./img/Figure_1a.png}
\caption{The reimplementation results.} \label{fig1}
\end{figure}
Our reimplementation, as the results confirm, work as expected and
reproduces the results obtained by Powers. We can observer that in our
reimplementation as well the individual type \textit{Cooperators +
  Small} win over the other species. We can see that, around
generation 20, the selfish large starts diminishing and both the
individuals with the small group gene start rising. Once the
\textit{Selfish + Large} species is extinct the \textit{Cooperators +
  Small} wins over the other small group individual type. The
\textit{Selfish + Large} individuals extinction is caused because,
after driving the \textit{Cooperators + Large} species extinct they
loose the advantage of consuming the cooperators resources. Meanwhile
between the small species the cooperators win over the cheaters since
they have an advantage when group size is 4 and the generations spent
reproducing within the group is 4.


\section{Extension}
In this section we consider an extension to the Powers' research. In
the conclusion section Powers states that future research should
consider applying the process of niche construction to other
parameters as well. We decided to apply niche construction to $t$, the
time spent for the within group reproduction step. We want to test
whether the original model environmental assumption about $t$ could
have played a role in facilitating a group proliferation over the
others.

\subsection{Design}
The first step was to modify Powers model in order to meet our new
research question. We introduced a new gene that expresses the
preferred time value $t$ for within group reproduction. The
initialization phase changes in such a way that all the possible
individuals types, according to this new gene, have the same equal
distribution. During the aggregation we draw groups from random
individuals specifying the same group size and time gene. The
dispersal and the normalization phase remain the same as in the
original model.

\subsection{Implementation}
We implemented this extension part by modifying the Main class of the
existing Java sources written to test Powers' research
reproducibility (see code at appendix \ref{mainext}). We used the
exact same parameters describe in table \ref{tab_parameters}, with the
exception of the $t$ value. For testing
purposes, we limited $t$ to take only two different values per
execution. This means that every execution will consist of eight species,
four with $t_1$ value and four with $t_2$ value for the $t$ gene. We
implemented two different migrant pools according to the $t$
gene. This, however, does not affect the results since the
normalization phase rescales the distribution like there is only one
population. 

\subsection{Results}
In this section we analyze the results obtained by testing different
values for the $t$ gene against each other. We first consider the case
when $t_1 = 4$ and $t_2 = 5$. Figures \ref{figt4t5A} and \ref{figt4t5B}
summarize the results. We can observe that giving the same parameters
we obtain two different results. This happens because of the
randomness of the aggregation phase. Since reproduction is applied
only within groups, how these groups are formed can influence the
number of the species' offspring. The important data to analyze,
however, is not only the genotype that reaches fixation, but how the small
groups of the $t5$ are behaving. We can see that in both cases the
\textit{Selfish + Small + $t_5$} drive the \textit{Cooperative +
Small + $t_5$} type extinct. We can see the same pattern in figures
\ref{figt2t3} and \ref{figt9t8}. When $t <= 4$ in small groups the
cooperators are favoured, but when $t > 4$ the cheaters have an higher
growth in the small groups. This indicates that the fixed
environmental value of $t$ in Powers research played a role in
facilitating the small cooperators to emerge against the small selfish
individuals. Furthermore, we can see the model limitations in the fact
that, as seen in all the aforementioned figures (\ref{figt4t5A},
\ref{figt4t5B}, \ref{figt2t3}, \ref{figt9t8}), applying only
reproduction between groups facilitates the individuals with a larger
$t$ value.

\begin{figure}
\includegraphics[width=\textwidth]{./img/Figure_t4t5A.png}
\caption{Population with $t_1 = 4$ and $t_2 = 5$.} \label{figt4t5A}
\end{figure}


\begin{figure}
\includegraphics[width=\textwidth]{./img/Figure_t4t5B.png}
\caption{Population with $t_1 = 4$ and $t_2 = 5$.} \label{figt4t5B}
\end{figure}

\begin{figure}
\includegraphics[width=\textwidth]{./img/Figure_t2t3.png}
\caption{Population with $t_1 = 2$ and $t_2 = 3$.} \label{figt2t3}
\end{figure}

\begin{figure}
\includegraphics[width=\textwidth]{./img/Figure_t9t8.png}
\caption{Population with $t_1 = 9$ and $t_2 = 8$.} \label{figt9t8}
\end{figure}

\section{Conclusion}
In the extension part we have investigated whether applying the
niche construction process in regards to the $t$ parameter, specifying the time
spent for within group reproduction, could reveal some environmental
assumption about the model. We discovered that the model described by
Powers works as expected only when $t <= 4$. We also observed the
model's limits regarding the reproduction step. Applying reproduction
only in between groups causes the species with a larger $t$ trait to
reach fixation.

%
% ---- Bibliography ----
%
% BibTeX users should specify bibliography style 'splncs04'.
% References will then be sorted and formatted in the correct style.
%
% \bibliographystyle{splncs04}
% \bibliography{mybibliography}
%
\begin{thebibliography}{8}

\bibitem{groups}
Powers, S.T., Penn, A.S. and Watson, R.A., 2007, September. Individual
selection for cooperative group formation. In European Conference on
Artificial Life (pp. 585-594). Springer, Berlin, Heidelberg.

\bibitem{simon}
Simon T. Powers, Social nice construction: evolutionary explanations for cooperative group forma-tion, University of Southampton, School of Electronics and Computer Science, Doctoral Thesis, 186pp, (2010)
  
\bibitem{b1}
Pfeiffer, T., Schuster, S., Bonhoeffer, S.: Cooperation and competition in the evolution of ATP-producing pathways. Science 292(5516) (2001) 504–507 

\bibitem{b2}
Pfeiffer, T., Bonhoeffer, S.: An evolutionary scenario for the
transition to undif-ferentiated multicellularity. PNAS 100(3) (2003)
1095–1098

\bibitem{b3}
Kreft, J.U.: Biofilms promote altruism. Microbiology 150 (2004)
2751–2760

\bibitem{b4}
Kreft, J.U., Bonhoeffer, S.: The evolution of groups of cooperating
bacteria and the growth rate versus yield trade-off. Microbiology 151
(2005) 637–641

\bibitem{b5}
Nowak, M.A., May, R.M.: The spatial dilemmas of
evolution. International Journal of Bifurcation and Chaos 3(1) (1993)
35–78

\bibitem{b6}
Wilson, D.S.: A theory of group selection. PNAS 72(1) (1975) 143–146 

\bibitem{b7}
Wilson, D.S.: The Natural Selection of Populations and Communities. Ben-jamin/Cummings (1980) 

\bibitem{b8}
Sober, E., Wilson, D.S.: Unto Others: The Evolution and Psychology of Unselfish Behavior. Harvard University Press, Cambridge, MA (1998) 

\bibitem{b9}
Wilson, D.S., Dugatkin, L.A.: Group selection and assortative
interactions. The American Naturalist 149(2) (1997) 336–351

\bibitem{b10}
Odling-Smee, F.J., Laland, K.N., Feldman, M.W.: Niche construction:
the ne-glected process in evolution. Monographs in population biology;
no. 37. Princeton University Press (2003)

\bibitem{b11}
Keller, L., ed.: Levels of Selection in Evolution. Monographs in behavior and ecology. Princeton University Press (1999)

\end{thebibliography}

\section{Appendix A: source code}\label{appendix}
This appendix section contains the source code of the program:
\begin{itemize}
\item Compile with: \textit{javac *.java}
\item Run with: \textit{java Main}
\item To plot the results run with: \textit{java Main $|$ xargs python3 plotBacteria.py}
\end{itemize}

\subsection{Main.java}\label{main}
\lstinputlisting[language=Java]{./src/Main.java}

\subsection{Individual.java}\label{ind}
\lstinputlisting[language=Java]{./src/Individual.java}

\subsection{IndividualFrequencies.java}\label{indfre}
\lstinputlisting[language=Java]{./src/IndividualFrequencies.java}

\subsection{GroupFrequencies.java}\label{groupfre}
\lstinputlisting[language=Java]{./src/GroupFrequencies.java}

\subsection{Main.java (extended version)}\label{mainext}
\lstinputlisting[language=Java]{./src/MainExtended.java}

\subsection{plotBacteria.py}\label{plot1}
\lstinputlisting[language=Java]{./src/plotBacteria.py}

\subsection{plotBacteria.py (extended version)}\label{plot1}
\lstinputlisting[language=Java]{./src/plotBacteriaExtended.py}

%% \section{Appendix B: program output}\label{appendixB}
%% This appendix section contains a complete output of the program:
%\lstinputlisting{output}

\end{document}
